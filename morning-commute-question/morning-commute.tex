\documentclass[a4paper,11pt]{article}

\usepackage[english]{babel}
\usepackage{float}
\usepackage{graphicx}
\usepackage{amsmath,amsthm}
\usepackage{gensymb}
\usepackage{amssymb}
\usepackage[margin=2.cm]{geometry}
\usepackage{pstricks-add}	%for geometric diagrams
\usepackage{chemfig}	%for structural formulae
\usepackage{tabularx}	%for better tables
\usepackage{booktabs}	%for better tables
\usepackage[makeroom]{cancel}	%for cancelling lines
\usepackage{hyperref}	%hyperlinks
\usepackage{mathrsfs}
\usepackage{mathtools}
\usepackage{epigraph}	%quotes
\usepackage{lastpage}
\usepackage{multicol}	%column environments
\usepackage{fancyhdr}	%headers
\usepackage[at]{easylist}	%easy lists
\usepackage{wasysym}
\usepackage{wrapfig}	%wrap figures in text
\usepackage{subfig}		%subfigures
\usepackage{tikz}

\allowdisplaybreaks
\newcommand\numberthis{\addtocounter{equation}{1}\tag{\theequation}}
\setlength{\epigraphwidth}{7.7cm}
\setlength{\tabcolsep}{10pt}

% bracket group shorthands
\newcommand{\abs}[1]{\left|#1\right|}
\newcommand{\set}[1]{\left\{#1\right\}}

% common sets
\newcommand{\R}{\mathbb{R}}
\newcommand{\Cmplx}{\mathbb{C}}
\newcommand{\Q}{\mathbb{Q}}
\newcommand{\Z}{\mathbb{Z}}
\newcommand{\N}{\mathbb{N}}

% derivative shorthands
\newcommand{\diff}[2]{\frac{\mathrm{d}#1}{\mathrm{d}#2}}
\newcommand{\pdiff}[2]{\frac{\partial #1}{\partial #2}}
\newcommand{\ndiff}[3]{\frac{\mathrm{d}^{#3}#1}{\mathrm{d}#2^{#3}}}
\newcommand{\npdiff}[3]{\frac{\partial^{#3} #1}{\partial #2^{#3}}}

% theorem environments
\newtheorem*{definition*}{Definition}
\newtheorem*{lemma*}{Lemma}
\newtheorem{theorem}{Theorem}
\newtheorem*{theorem*}{Theorem}
\newtheorem*{corollary*}{Corollary}
\newtheorem{example}{Example}
\newtheorem*{remark}{Remark}

\DeclareMathOperator{\bdy}{Bdy}
\DeclareMathOperator{\interior}{Int}

% header
\pagestyle{fancy}
\lhead{Now, for something completely different...}
\rhead{Year 11 2018}

%\title{Morning commute question}
%\date{\today}
%\author{Daniel Czapski}

\begin{document}
\subsection*{A harder combinatorics question: inspired by my morning commute}
It is well known by frequent users of public transport that every carriage on a Sydney Trains train has a four digit serial number comprised of digits between 0 and 9 (inclusive!). The ``Train Game'' (or whatever students call it now) has the aim of combining the digits of the carriage serial number, using standard operations, to make ten\footnote{Sometimes this is very easy; other times this is particularly difficult and sometimes it is impossible! An interesting problem to consider is this: what conditions must be imposed on a serial number in order to guarantee that there exists at least one way to make ten? This is an extremely non-trivial problem!}. Traditionally, it is stipulated that no number can be used more than once and all numbers must be used in the order in which they appear. For example, 1345: $(1-3+4)\times5$.\\

\noindent I'm bad at basic arithmetic and my morning train ride isn't that long, so I prefer to rearrange the numbers into whatever order is convenient. Thus, suppose the order of the digits is NOT important (i.e. 1234 and 3421 are equivalent). \\
	\begin{easylist}[enumerate]
		\ListProperties(Numbers1=r,Numbers2=r,Progressive=1cm,Margin1=1cm,FinalMark={)},Space*=0.5cm)
		@ If serial numbers have no repetitions, how many different serial numbers are possible?
    \end{easylist}
    \vspace{0.5cm}
    \noindent Serial numbers often DO have repeats, however. For example, a three digit serial number could comprise a triple (e.g. 111), a pair and a single (e.g. 112) or three singles (e.g. 123); there are three unique cases to consider.\\
	\begin{easylist}[enumerate]
		\ListProperties(Start1=2,Numbers1=r,Numbers2=r,Progressive=1cm,Margin1=1cm,FinalMark={)},Space*=0.5cm)
        @ For a four digit serial number, how many unique cases are there? List them, with an example.
        @ How many different arrangements are possible, assuming order is unimportant (i.e. 1234 and 3421 are considered the same, as they both comprise 1, 2, 3 and 4) and repetitions are permitted?
        @ Repeat parts i), ii) and iii) for five digit serial numbers.
    \end{easylist}
    \vspace{0.5cm}
\noindent Suppose we insist that we use the numbers in order. That is, order is important (i.e. 1234 and 3421 are NOT equivalent). \\
	\begin{easylist}[enumerate]
		\ListProperties(Start1=5,Numbers1=r,Numbers2=r,Progressive=1cm,Margin1=1cm,FinalMark={)},Space*=0.5cm)
% 		@ Repeat parts i), iii) and iv) assuming order is important.
		@ Repeat the question assuming order IS important.
    \end{easylist}
    
% I reckon this question is quite challenging! I'm very interested to see what solutions are devised.

\end{document}