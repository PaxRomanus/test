\documentclass[a4paper,11pt]{article}

\usepackage[english]{babel}
\usepackage{float}
\usepackage{graphicx}
\usepackage{amsmath,amsthm}
\usepackage{gensymb}
\usepackage{amssymb}
\usepackage[margin=2.cm]{geometry}
\usepackage{pstricks-add}	%for geometric diagrams
\usepackage{chemfig}	%for structural formulae
\usepackage{tabularx}	%for better tables
\usepackage{booktabs}	%for better tables
\usepackage[makeroom]{cancel}	%for cancelling lines
\usepackage{hyperref}	%hyperlinks
\usepackage{mathrsfs}
\usepackage{mathtools}
\usepackage{epigraph}	%quotes
\usepackage{lastpage}
\usepackage{multicol}	%column environments
\usepackage{fancyhdr}	%headers
\usepackage[at]{easylist}	%easy lists
\usepackage{wasysym}
\usepackage{wrapfig}	%wrap figures in text
\usepackage{subfig}		%subfigures
\usepackage{tikz}

\allowdisplaybreaks
\newcommand\numberthis{\addtocounter{equation}{1}\tag{\theequation}}
\setlength{\epigraphwidth}{7.7cm}
\setlength{\tabcolsep}{10pt}

% bracket group shorthands
\newcommand{\abs}[1]{\left|#1\right|}
\newcommand{\set}[1]{\left\{#1\right\}}

% common sets
\newcommand{\R}{\mathbb{R}}
\newcommand{\Cmplx}{\mathbb{C}}
\newcommand{\Q}{\mathbb{Q}}
\newcommand{\Z}{\mathbb{Z}}
\newcommand{\N}{\mathbb{N}}

% derivative shorthands
\newcommand{\diff}[2]{\frac{\mathrm{d}#1}{\mathrm{d}#2}}
\newcommand{\pdiff}[2]{\frac{\partial #1}{\partial #2}}
\newcommand{\ndiff}[3]{\frac{\mathrm{d}^{#3}#1}{\mathrm{d}#2^{#3}}}
\newcommand{\npdiff}[3]{\frac{\partial^{#3} #1}{\partial #2^{#3}}}

% theorem environments
\newtheorem*{definition*}{Definition}
\newtheorem*{lemma*}{Lemma}
\newtheorem{theorem}{Theorem}
\newtheorem*{theorem*}{Theorem}
\newtheorem*{corollary*}{Corollary}
\newtheorem{example}{Example}
\newtheorem*{remark}{Remark}

\DeclareMathOperator{\bdy}{Bdy}
\DeclareMathOperator{\interior}{Int}

% header
\pagestyle{fancy}
\lhead{Locus, the Parabola and Parametrics}
\rhead{Year 11 2018}

%\title{Locus, the Parabola and Parametrics}
%\date{\today}
%\author{Daniel Czapski}

\begin{document}
	\subsection*{Question 1: 2001 Q6 b.}
	Consider the point $P(2at,at^2)$ on the parabola $x^2=4ay$.\\
    
	\begin{easylist}[enumerate]
		\ListProperties(Numbers1=r,Numbers2=a,Progressive=1cm,Margin1=1cm,FinalMark={)},Space*=0.5cm)
		@ Prove that the equation of the normal at $P$ is $x+ty=2at+at^3$.
        @ Find the coordinates of the point $Q$ on the parabola such that the normal at $Q$ is perpendicular to the normal at $P$.
        @ Show that the two normals of part ii) intersect at the point $R$ with coordinates 
        \begin{align*}
        x = a\left(t-\frac{1}{t}\right) && y = a\left(t^2+1+\frac{1}{t^2}\right).
        \end{align*}
        @ Find the Cartesian form of the equation of the locus of the point $R$.
	\end{easylist}
    
    
    \subsection*{Question 2: 2014 Q13 c.}
    The point $P(2at,at^2)$ lies on the parabola $x^2=4ay$ with focus $S$. The point $Q$ divides the interval $PS$ internally in the ratio $t^2:1$.\\
    
	\begin{easylist}[enumerate]
		\ListProperties(Numbers1=r,Numbers2=a,Progressive=1cm,Margin1=1cm,FinalMark={)},Space*=0.5cm)
		@ Show that the coordinates of the point $Q$ are given by 
        \begin{align*}
        x=\frac{2at}{1+t^2} && y=\frac{2at^2}{1+t^2}.
        \end{align*}
        @ Express the slope of $OQ$ in terms of $t$.
        @ Using the result from part ii), or otherwise, show that $Q$ lies on a fixed circle of radius $a$.
	\end{easylist}
    \pagebreak

\end{document}
